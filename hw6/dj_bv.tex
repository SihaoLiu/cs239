% \documentclass[preview]{standalone}
% If the image is too large to fit on this documentclass use
\documentclass[draft]{beamer}
% img_width = 13, img_depth = 12
\usepackage[size=custom,height=19,width=174,scale=0.7]{beamerposter}
% instead and customize the height and width (in cm) to fit.
% Large images may run out of memory quickly.
% To fix this use the LuaLaTeX compiler, which dynamically
% allocates memory.
\usepackage[braket, qm]{qcircuit}
\usepackage{amsmath}
\pdfmapfile{+sansmathaccent.map}
% \usepackage[landscape]{geometry}
% Comment out the above line if using the beamer documentclass.
\begin{document}
\begin{equation*}
    \Qcircuit @C=1.0em @R=0.0em @!R {
                \lstick{ {q}_{0} :  } & \gate{X} & \gate{H} & \multigate{6}{unitary} & \qw & \qw & \qw & \qw & \qw & \qw & \qw & \qw & \qw\\
                \lstick{ {q}_{1} :  } & \gate{H} & \qw & \ghost{unitary} & \gate{H} & \meter & \qw & \qw & \qw & \qw & \qw & \qw & \qw\\
                \lstick{ {q}_{2} :  } & \gate{H} & \qw & \ghost{unitary} & \gate{H} & \qw & \meter & \qw & \qw & \qw & \qw & \qw & \qw\\
                \lstick{ {q}_{3} :  } & \gate{H} & \qw & \ghost{unitary} & \gate{H} & \qw & \qw & \meter & \qw & \qw & \qw & \qw & \qw\\
                \lstick{ {q}_{4} :  } & \gate{H} & \qw & \ghost{unitary} & \gate{H} & \qw & \qw & \qw & \meter & \qw & \qw & \qw & \qw\\
                \lstick{ {q}_{5} :  } & \gate{H} & \qw & \ghost{unitary} & \gate{H} & \qw & \qw & \qw & \qw & \meter & \qw & \qw & \qw\\
                \lstick{ {q}_{6} :  } & \gate{H} & \qw & \ghost{unitary} & \gate{H} & \qw & \qw & \qw & \qw & \qw & \meter & \qw & \qw\\
                \lstick{c_{0}: } & \cw & \cw & \cw & \cw & \cw \cwx[-6] & \cw & \cw & \cw & \cw & \cw & \cw & \cw\\
                \lstick{c_{1}: } & \cw & \cw & \cw & \cw & \cw & \cw \cwx[-6] & \cw & \cw & \cw & \cw & \cw & \cw\\
                \lstick{c_{2}: } & \cw & \cw & \cw & \cw & \cw & \cw & \cw \cwx[-6] & \cw & \cw & \cw & \cw & \cw\\
                \lstick{c_{3}: } & \cw & \cw & \cw & \cw & \cw & \cw & \cw & \cw \cwx[-6] & \cw & \cw & \cw & \cw\\
                \lstick{c_{4}: } & \cw & \cw & \cw & \cw & \cw & \cw & \cw & \cw & \cw \cwx[-6] & \cw & \cw & \cw\\
                \lstick{c_{5}: } & \cw & \cw & \cw & \cw & \cw & \cw & \cw & \cw & \cw & \cw \cwx[-6] & \cw & \cw\\
         }
\end{equation*}

\end{document}